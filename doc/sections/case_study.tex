
\subsection{Behavioural classification}


\subsection{prey capture identification}

In order to identify 

\begin{itemize}
    \item We included all dives deeper than 30 meters, which is in line with \citet{Wright:2017}
    %
    \item We compared with Tennessen's method
    %
    \item we categorized a segment as a ``crunch" associated with prey capture if a skilled researcher confirmed a crunching noise at the end of the bottom phase of a dive or if a ``crunch" noise was heard on the ascent phase of the dive. This is in also in line with \citet{Wright:2017}, who found that killer whales tend to catch Chinook at depth and then immediate come to the surface. The full dive was then categorized as a successful foraging dive.
    %
    \item we categorized a dive as ``negative" if the camera was on for the entire dive and no signs of prey capture were observed. Signs of foraging include crunching noises, visible scales, or confirmed visible prey capture by the killer whale.
    %
    \item all other dives below 30 m were categorized as ``unknown".
\end{itemize}
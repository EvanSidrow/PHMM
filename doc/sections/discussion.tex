% !TeX root = ../main.tex

%As ecologists continue to collect larger and more complicated biologging data sets, the HMMs used to model them have also become more sophisticated \citep{Adam:2019, Lawler:2019, Sidrow:2021}. However, some complicated animal behaviours remain difficult to decode using biologging data alone. Labelled data can improve predictive performance \citep{Krogh:1997}, but in ecology, labels can be prohibitively difficult or expensive to collect in large quantities. Several studies have incorporated labelled observations into HMMs \citep{McClintock:2012, Pirotta:2018}, but the influence of the labelled data can be negligible when there is very little unlabelled data. This issue often affects biologically meaningful behaviours that are critical for conservation efforts, including foraging \citep{Saldanha:2023}. 

In this chapter, I incorporated sparse labels into an HMM in a natural way that changes the influence of unlabelled observations and demonstrably improves predictive performance. In particular, I weighted observations without associated labels within the likelihood of the HMM using a parameter $\alpha \in [0,1]$. On the extremes, $\alpha = 0$ corresponds to an HMM which totally ignores unlabelled data and $\alpha = 1$ corresponds to a traditional, unweighted HMM. Practitioners can use cross-validation with their metric of choice to find an optimal value of $\alpha$.

In addition to better performance, my method has two additional benefits compared to previous methods. First, it estimates the probability that each dive is a successful foraging dive instead of outputting a binary label. This allows practitioners to set a threshold to control the false positive and false negative rates. Second, my method also categorizes subdive states so practitioners can simultaneously categorize the subdive behaviour of the killer whale while they estimate successful foraging dives.

Future work can quantify exactly how the density of labels and level of model misspecification affects the value of $\alpha$ that optimizes prediction accuracy and model fit. The biological interpretability and predictive accuracy of the model from the first case study were optimized for $\alpha \approx 0.05$. In the second case study, the model had better predictive performance for $\alpha = 0.01$, and each PHMM had distinct biological interpretations for $\alpha \leq 0.01$ and $\alpha > 0.01$. The ``bottom" and ``chase" subdive states were better separated for the PHMMs with $\alpha > 0.01$, but not necessarily in a way that was biologically natural. It is intuitive that subdive states with no labels were better separated within PHMMs that weighted unlabelled observations more heavily. Quantifying the precise relationship between $\alpha$ and the state-dependent distributions is a promising direction for future research. 

As is common in movement ecology, my case studies had very few labels, which can negatively impact the reliability of cross-validation. Therefore, future work can apply this weighted likelihood approach to case studies with more labels, potentially from other fields. I also performed case studies where subject matter experts were confident in their labels. Future work can focus on how including labels affects the performance of the PHMM when practitioners are less confident in their labels and infer $\bfbeta$ as a parameter of the PHMM.

Although I did not have a large enough sample size to draw definitive conclusions, the preliminary results from my method are in line with those from \citet{Tennessen:2023}, who conducted a comprehensive study of foraging behaviour in northern and southern resident killer whales. In particular, my results support their findings that northern resident killer whales tend to spend more time travelling and resting compared to southern residents, and that the northern residents tend to catch more fish per unit of effort compared to the southern residents. However, I found lower numbers of prey captures per hour for both northern and southern residents compared to \citet{Tennessen:2023}. While the exact cause is unclear, the discrepancy could be due to my small sample size or limitations where only certain kinds of prey capture events were labelled in my data set.

We primarily focus on identifying the behaviour of killer whales in this work, but incorporating sparse labels into complex HMMs is a common modelling problem across a variety of use cases and disciplines. In addition, complicated time series data are increasingly common as sensing technology continues to improve \citep{Patterson:2017}. As such, the modelling approach developed here can help researchers effectively model complicated, sparsely labelled time series to optimize prediction accuracy and model fit.